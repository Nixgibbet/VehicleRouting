%%%%%%%%%%%%%%%%%%%%%%%%%%%%%%%%%%%%%%%%%
% Beamer Presentation
% LaTeX Template
% Version 1.0 (10/11/12)
%
% This template has been downloaded from:
% http://www.LaTeXTemplates.com
%
% License:
% CC BY-NC-SA 3.0 (http://creativecommons.org/licenses/by-nc-sa/3.0/)
%
%%%%%%%%%%%%%%%%%%%%%%%%%%%%%%%%%%%%%%%%%

%----------------------------------------------------------------------------------------
%	PACKAGES AND THEMES
%----------------------------------------------------------------------------------------

\documentclass{beamer}
\usepackage{wrapfig}
\usepackage{tikz}
\def\firstcircle{(0,0) circle (1cm)}
\def\secondcircle{(60:1.5cm) circle (1cm)}
\def\thirdcircle{(0:1.5cm) circle (1cm)}
\mode<presentation> {

% The Beamer class comes with a number of default slide themes
% which change the colors and layouts of slides. Below this is a list
% of all the themes, uncomment each in turn to see what they look like.

%\usetheme{default}
%\usetheme{AnnArbor}
%\usetheme{Antibes}
%\usetheme{Bergen}
%\usetheme{Berkeley}
%\usetheme{Berlin}
%\usetheme{Boadilla}
%\usetheme{CambridgeUS}
%\usetheme{Copenhagen}
\usetheme{Darmstadt}
%\usetheme{Dresden}
%\usetheme{Frankfurt}
%\usetheme{Goettingen}
%\usetheme{Hannover}
%\usetheme{Ilmenau}
%\usetheme{JuanLesPins}
%\usetheme{Luebeck}
%\usetheme{Madrid}
%\usetheme{Malmoe}
%\usetheme{Marburg}
%\usetheme{Montpellier}
%\usetheme{PaloAlto}
%\usetheme{Pittsburgh}
%\usetheme{Rochester}
%\usetheme{Singapore}
%\usetheme{Szeged}
%\usetheme{Warsaw}

% As well as themes, the Beamer class has a number of color themes
% for any slide theme. Uncomment each of these in turn to see how it
% changes the colors of your current slide theme.

%\usecolortheme{albatross}
%\usecolortheme{beaver}
%\usecolortheme{beetle}
%\usecolortheme{crane}
%\usecolortheme{dolphin}
%\usecolortheme{dove}
%\usecolortheme{fly}
%\usecolortheme{lily}
%\usecolortheme{orchid}
%\usecolortheme{rose}
%\usecolortheme{seagull}
%\usecolortheme{seahorse}
%\usecolortheme{whale}
%\usecolortheme{wolverine}

%\setbeamertemplate{footline} % To remove the footer line in all slides uncomment this line
\setbeamertemplate{footline}[page number] % To replace the footer line in all slides with a simple slide count uncomment this line

%\setbeamertemplate{navigation symbols}{} % To remove the navigation symbols from the bottom of all slides uncomment this line
}
\usepackage[simplified]{pgf -umlcd}
\renewcommand{\umltextcolor}{black}
\renewcommand{\umldrawcolor}{black}
\renewcommand{\umlfillcolor}{white}
\usepackage{multicol}
\usepackage{adjustbox}
\usepackage{tikz}
\usetikzlibrary[topaths]
\usetikzlibrary{positioning,chains,fit,shapes,calc,graphs,graphs.standard,automata}
\usepackage[utf8]{inputenc} 
\usepackage[T1]{fontenc}
\usepackage{listings}
\usepackage[german]{babel} 
\usepackage{adjustbox}
\usepackage{graphicx} % Allows including images
\usepackage{booktabs} % Allows the use of \toprule, \midrule and \bottomrule in tables

%----------------------------------------------------------------------------------------
%	TITLE PAGE
%----------------------------------------------------------------------------------------

\title[VRP]{Vehicle Routing Problem} % The short title appears at the bottom of every slide, the full title is only on the title page

\author{Michael Renn und Tobias Krieger} % Your name
\institute[H-BRS] % Your institution as it will appear on the bottom of every slide, may be shorthand to save space
{
Hochschule Bonn-Rhein-Sieg \\ % Your institution for the title page
\medskip
\textit{Michael.Renn@smail.inf.h-brs.de}
\textit{Tobias.Krieger@smail.inf.h-brs.de} % Your email address
}
\date{November 23, 2019} % Date, can be changed to a custom date



\begin{document}
\begin{frame}
\titlepage % Print the title page as the first slide
\end{frame}

\begin{frame}
\frametitle{Inhaltsverzeichnis} % Table of contents slide, comment this block out to remove it
\tableofcontents % Throughout your presentation, if you choose to use \section{} and \subsection{} commands, these will automatically be printed on this slide as an overview of your presentation
\end{frame}

%----------------------------------------------------------------------------------------
%	PRESENTATION SLIDES
%----------------------------------------------------------------------------------------

%------------------------------------------------
\section{Einstieg} % Sections can be created in order to organize your presentation into discrete blocks, all sections and subsections are automatically printed in the table of contents as an overview of the talk
%------------------------------------------------

 % A subsection can be created just before a set of slides with a common theme to further break down your presentation into chunks


\subsection{Was sind VRP} 
\begin{frame}
\frametitle{Idee VRP}

\end{frame}
\subsection{Wofür brauch man VRP}
\begin{frame}
\frametitle{Klassiches VRP Problem}
\begin{columns}[T] % align columns
\begin{column}{.48\textwidth}
\centering
\begin{tikzpicture}

\def \n {5}
\def \radius {3cm}
\def \margin {8} % margin in angles, depends on the radius

\foreach \s in {1,...,\n}
{
  \node[draw, circle] at ({360/\n * (\s - 1)}:\radius) {$\s$};
  \draw[->, >=latex] ({360/\n * (\s - 1)+\margin}:\radius) 
    arc ({360/\n * (\s - 1)+\margin}:{360/\n * (\s)-\margin}:\radius);
}
\end{tikzpicture} 
    \caption{Klassisches Problem grafisch Dargestellt} \label{CIA}
\end{column}%
\hfill%


\end{columns}
\end{frame}
%------------------------------------------------
%------------------------------------------------
\section{Grundlagen}
\subsection{Mathe \& Graphen}
\begin{frame}
\frametitle{Mathe}

\end{frame}

\subsubsection*{}
\begin{frame}{Graphen}

\end{frame}

\subsection{Heuristik}
\begin{frame}{Theorie}
\end{frame}
\begin{frame}{Beispiel}
\end{frame}


\subsection{Metaheuristik}
\begin{frame}{Theorie}

\end{frame}
\begin{frame}{Beispiel}
\end{frame}
%------------------------------------------------
%------------------------------------------------
\section{Unsere Idee}
\begin{frame}{Anwendungsgebiete von VRP}

\end{frame}
\subsection{Artifical Insects}
\begin{frame}{Bee}

\end{frame}
\begin{frame}{Ant}

\end{frame}
\subsection{Realisierung}
\begin{frame}{Moa Framework}

\end{frame}
%------------------------------------------------
%------------------------------------------------
\section{Rollback \& Conclusion}
\begin{frame}{Was bisher geschah}

\end{frame}
\begin{frame}{Conclusion}
    
\end{frame}
%------------------------------------------------
%------------------------------------------------
\section{Literatur}
\begin{frame}[allowframebreaks]
\frametitle{Literatur}
\footnotesize{
\begin{thebibliography}{99} % Beamer does not support BibTeX so references must be inserted manually as below

%\bibitem[Blakley, 1979]{blak}G. R. Blakley 
%\newblock {Safeguarding cryptographic keys}
%\newblock \emph {Proceedings of the national computer conference} 


%\bibitem[Shamir, 1979]{b8}A. Shamir 
%\newblock{How to Share a Secret}
%\newblock \emph{Commun. ACM}


%\bibitem[Benaloh, 1987]{b1} J. C. Benaloh
%\newblock {Secret Sharing Homomorphisms: Keeping Shares of a Secret Secret (Extended Abstract)}
%\newblock \emph {Advances in Cryptology --- CRYPTO' 86}

%\bibitem[Büscher et al., 2017]{Background}N. Büscher und S. Katzenbeisser
%\newblock{Background}
%\newblock \emph {Compilation for Secure Multi-party Computation}

%\bibitem[Nojoumian, 2012]{bild}M. Nojoumian \newblock{Novel Secret Sharing and Commitment Schemes for Cryptographic Applications}

%\bibitem[Yao, 1986]{Yao2}A. C. Yao
%\newblock{How to generate and exchange secrets} 
%\newblock \emph {27th Annual Symposium on Foundations of Computer Science}

%\bibitem[Goldreich et al., 1987]{GMW} O. Goldreich et al.
%\newblock{How to play any mental game} 
%\newblock \emph {Proceedings of the nineteenth annual ACM symposium on Theory of computing}


%\bibitem[Choi et al., 2012]{Gmwprot}S. G. Choi et al.
%\newblock{Secure multi-party computation of boolean circuits with applications to privacy in on-line marketplaces}
%\newblock \emph{Cryptographers’ Track at the RSA Conference}

%\bibitem[Crépeau et al., 2002]{Quantum}C. Crépeau et al.
%\newblock{ Secure multi-party quantum computation}
%\newblock \emph{Proceedings of the thiry-fourth annual ACM symposium on Theory of computing}


%\bibitem[Wootters et al., 1982]{noclone} W. K. Wootters et al.
%\newblock{A single quantum cannot be cloned}
%\newblock \emph{Nature 299}

%\bibitem[Cleve et al., 1999]{QuantumSec} R. Cleve et al.
%\newblock{How to share a quantum secret}
%\newblock\emph{Physical Review Letters 83}

%\bibitem[Bogdanov, 2013]{share}D. Bogdanov
%\newblock{Sharemind: programmable secnre computations with practical applications} 
%\newblock \emph{PhD Thesis, University of Tartu}

%\bibitem[Archer et al., 2018]{b7} D. W. Archer et al.
%\newblock{From Keys to Databases—Real-World Applications of Secure Multi-Party Computation}, \newblock \emph{The Computer Journal}

%\bibitem[Nikova et al., 2006]{channel}S. Nikova et al.
%\newblock{Threshold implementations against side-channel attacks and glitches}
%\newblock \emph{International conference on information and communications security}

%\bibitem[Unruh, 2018]{b10} D. Unruh \newblock{Everlasting Multi-party Computation} \newblock\emph{Journal of Cryptography}

\end{thebibliography}
}
\end{frame}
%------------------------------------------------
%------------------------------------------------

\section*{}
\begin{frame}
    \centering 
   \huge Vielen Dank für Ihre Aufmerksamkeit!
\end{frame}
%----------------------------------------------------------------------------------------

\end{document}
